\newpage

\section{Control Model}


\subsection{Position target : what we want to do}

Controlling the behaviour of the machine, in the less restrictive meaning, means determining successive actuation
distances, and time intervals between them.\newline

Ideally, we would like to determine consecutive positions (in any of the two coordinate systems), then determine
actuation distances, and determine time intervals so that speed of a group of axis (in any of the two coordinate
systems) matches a target one;\newline


\subsection{Physical limitations : what we can do}


However, this is not so simple. Indeed, the fact of determining target distances for each actuator, at a given
state, will by itself impose constraints on the time;\newline

Suppose the machine in the state S. We wish to execute a movement, whose actuation distances have been computed.\newline

Actuators are at a determined speed, and still have physical limitations. It is important to mention here that
actuators are all moved simultaneously.\
    Therefore, the time we choose for this movement will determine actuators speeds. But as we said previously,
the variation of actuators speeds can't be set randomly without risking to damage actuators and the machine;\newline

More formally, given a machine state S and a set of actuation distances $(d_i)_i$, each actuator will provide a
duration interval $I_i = [t_{i_{\min}}, t_{i_{\max}}]$ that is acceptable for it;\newline

Given this set of intervals, we must choose a duration $t$ that respects all time constraints, ie
$d \in \bigcap \limits _{i \in E} I_i$\newline

What now if $\bigcap \limits _{i \in E} I_i = \emptyset$ ?\newline

This means that the movement we are trying to execute is simply non realisable. This is caused by two time
intervals that do not intersect, ie two actuators, that have incompatible speed variations : there is no compatible
duration for one to decelerate slowly enough, and for the other to accelerate slowly enough;
\footnote{This sentence can lead to assume erroneously that speed limitations only reside in their derivative,
which might be true in simple machines, but in more complex ones (robotic arms for ex), may not be always the
case.}\newline

There are two potential solutions to this problem.\newline

The naive (but simple !) solution, is to select an arbitrary time, given an absolute criteria. We could for example
select a time that would limit acceleration, but break deceleration rules. The two great advantages of this
solution are to respect required movement distances, and to require few calculation; The disavantage is that it
breaks actuations limitations\newline

The second solution complements the first one by modifying movement distances to match actuators limitations.
Now that we have determined our movement time, we will modify movement distance for actuators that see their
limitations broken; This solution makes the assumption that the movement distance that matched speed limitations
for a given time can actually be determined.\
    If so, this method will give a movement that can be safely executed by all actuators in its duration.\newline

For now, we have determined movement distances, and a time interval in which we can safely choose the movement
duration. This interval can be of length 0, and then we have no choice on the duration;\newline


\newpage

\subsection{Kinematic constraints : what we choose to do}

Now that we have a determined set of movement distances, and a variable duration, we can select the most relevant
duration.\newline

We could, for example, restrict again the duration interval, to values that match acceleration bounds on a control
axis.
It could also be possible to verify that a deceleration is required to satisfy eventual jerk bounds on a control
axis.\newline

More generally, the selection of the duration comes by consecutive restrictions of the duration interval by ordered
optional constraints;
These constraints are functions that take the current state of the machine and provide a duration bound.
They are qualified optionnal because the final duration will necessarily be in the interval that satisfies actuators
constraints;
They are qualified ordered because a constraint will evaluated only if the interval is not of length 0;\newline

There can be as many constraints as we want. They are evaluated in order until interval is of length 0, or until
all have been evaluated;
If all constraints have been evaluated and the duration interval is not of length 0, the upper duration is selected;
\newline

