
\section{Planner}

This section describes the planner algorithm implemented in the project.


A planner's purpose is, given a set of trajectories that share the same dimension, to generate a sequence of points,
belonging (or not) to those trajectories


In order to describe its main objectives more accurately, we need to elaborate a bit on the notion of trajectory;


\subsection{Trajectory}

Definition : a trajectory t is a continuous application from a segment of R in $R^n$ :\newline

$t : [a, b] -> R^n$, $a$ and $b$ two real numbers, $n$ a positive integer, t continuous;

Naming convention : 
\begin{itemize}
\item [-] n : the dimension of the trajectory;
\item [-] x in [a, b] : an index of the trajectory;
\item [-] t(x), x in [a, b] : a point of the trajectory;
\item [-] a : the minimal index, the initial index;
\item [-] t(a) : the initial point;
\item [-] b : the maximal index, the final index;
\item [-] t(b) : the final point;\newline
\end{itemize}

A trajectory $t$ represents a path in a particular space from its initial point $t_i$ to its final point $t_f$.

Its aim is to fully determine the path that some system must follow in this space;


\subsection{Hook mode}

Let us consider now, two trajectories, $u$ and $v$ that share the same dimension, and must be followed one after the
other, $u_i$, $u_f$, $v_i$, $v_f$ initial and final points of $u$ and $v$.

If both trajectories are joint, ie $u_f = v_i$, the path between $u_i$ and $v_f$ is fully determined.

If trajectories are disjoint, the path is simply undetermined, as there is no trajectory defined between
$u_f$ and $ v_i$to do so.

A planner can accept only a sequence of consecutively joint trajectories, or can accept any sequence of trajectories.

In the second case, it must contain a hook algorithm, that is in charge of determining an arbitrary path
from $u_f$ to $ v_i$.

The planer has to work modes, trajectory mode, where it processes a trajectory, and hook mode, where it attempts
to reach a trajectory's initial point;


\subsection{Transition point}

A transition point is a point where a brutal change of direction occurs.

As a planer can be used to control physical systems, that may suffer of this kind of changes, it may be important to
monitor the existence of those points.

A planner can offer this monitoring, so that actions can be taken by another piece of code;

The presence of transition points inside trajectories is uncertain, and must be determined manually;

A planner can also offer this functionality, but it involves traversing the trajectory entirely, and it can be heavy
for some low-performance systems;

However if we cannot easily detect transition points inside a trajectory, there are two immediate transition points,
at extremal points.

When completing a trajectory (reaching its final point), the planner may, enter into hook mode.
The hook algorithm is used to determine a path to the next trajectory's initial point.
This is the first transition point, as the direction the planner takes may not be the direction at the end of the
trajectory;\newline

When the hook algorithm manages to reach the next trajectory's initial point, the planner goes into trajectory mode,
and starts searching the trajectory for an adequate point.
This is the second transition point, as the direction the next trajectory takes may not be the direction the planner
took to reach its initial point.\newline


\subsection{Managing transition points}

The planner lists transition points in their order of appearance in the global path;

When the planner has processed all trajectories, the list is empty;
When the planner is processing a trajectory, the first transition point in the list corresponds to the current
trajectory's final point;
When the planner is hooking the next trajectory, the first transition point in the list corresponds to the initial
point of the trajectory to hook;

This list is updated the following way :

When adding a trajectory to the planner's list, the planner determines if the trajectory is joint to its
predecessor.\newline

If so, it modifies the last transition point, to reflect the direction change due to the trajectory, and marks the
transition to trigger trajectory mode;\newline

If not, it modifies the last transition point, to reflect the direction change due to the hook algorithm, and mark it
to trigger hook mode;
it then adds a transition point to monitor, at the initial point of the new trajectory, to reflect the direction
change between the hook aglorithm and the trajectory;

Then, it attempts to find transition points in the trajectory.
If it finds some, it adds them in their order of appearance;
As stated earlier, this operation is time-consuming, and can be skipped.

Finally, it creates another transition point to monitor, representing the stop at the end of the new trajectory.
This transition will be modified, when a new trajectory will be added, as described just before, to reflect a direction
change instead of a stop.

When processing a trajectory, when a transition point is reached inside the trajectory, the planner removes the first
transition point of the list;

At trajectory mode exit, when the planner reaches the last point of a trajectory, or at hook mode exit, when the
planner has hooked the first point of a trajectory, it removes the first transition point, and depending on its type,
it enters into hook mode, or trajectory mode;


\subsection{Trajectory mode}

%TODO